\documentclass[a5paper,10pt]{article}
\usepackage[top=1cm,bottom=1cm,left=1cm,right=1cm]{geometry}
\usepackage{graphicx}
\usepackage{xcolor}
\usepackage{titlesec}
\definecolor{recipecolor}{RGB}{85,107,47}
\titleformat{\section}{\Large\bfseries\color{recipecolor}}{}{0em}{}
\titleformat{\subsection}{\bfseries\color{recipecolor}}{}{0em}{}
\setcounter{secnumdepth}{0}

\begin{document}

\begin{center}
    \textbf{\huge \color{recipecolor} High Protein Big Mac Rolle}\\
    \vspace{0.5cm}
    \textit{Chrissi's Lieblingsrezept. Gibt es immer, wenn man Chrissi was Gutes tun will!}
\end{center}

\noindent2 Portionen

\vspace{0.5cm}
\noindent\textbf{\large Zutaten}
\vspace{0.2cm}

\noindent\textbf{Teig:}
\begin{itemize}
    \item 250g Magerquark
    \item 120g geriebener Käse light
    \item 3 Eier
    \item Salz, Pfeffer
\end{itemize}

\noindent\textbf{Füllung:}
\begin{itemize}
    \item 300g Rinderhackfleisch, eventuell fettreduziert
    \item 1/2 Zwiebel
    \item 3 Scheiben Schmelzkäse
    \item 4 saure Gurken
    \item 2 Handvoll geschnittener Eisbergsalat
    \item 1 Tomate
    \item Salz, Pfeffer
    \item 1 TL Rapsöl
\end{itemize}

\noindent\textbf{Soße:}
\begin{itemize}
    \item 3 EL Skyr
    \item 1 EL saure Sahne / Sauerrahm
    \item 2 EL Ketchup
    \item 1 TL Senf
    \item 1/2 TL Weißweinessig
    \item Etwas Essigwasser von den Gurken
\end{itemize}

\newpage
\noindent\textbf{\large Zubereitung}
\begin{enumerate}
    \item Eier, Quark und Käse verrühren. Mit etwas Salz und Pfeffer würzen.
    \item Die Masse auf ein mit Backpapier belegtes Backblech streichen und bei 180°C Ober- / Unterhitze ca. 20 Minuten backen bis der Teig ein wenig gebräunt ist.
    \item Für die Füllung die Zwiebel und das Hackfleisch in Öl anbraten. Mit Salz und Pfeffer würzen.
    \item Das Gemüse in feine Scheiben schneiden.
    \item Für die Soße alle Zutaten verrühren und mit etwas Salz und Pfeffer abschmecken.
    \item Den Teig vom Blech nehmen und das Backpapier abziehen. Mit der Soße den Teig bestreichen und dabei etwas übrig lassen.
    \item Nun das Hackfleisch auf einem Drittel des Bodens verteilen, den Schmelzkäse darüber legen, damit er etwas schmelzen kann. Dann die restlichen Zutaten auf dem Teigboden verteilen.
    \item Die restliche Soße darauf geben und dann vorsichtig aufrollen.
    \item Zum Servieren in 2 Teile schneiden und genießen.
\end{enumerate}

\end{document}
